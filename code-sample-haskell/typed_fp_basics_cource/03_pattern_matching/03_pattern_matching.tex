% texstudio
\documentclass[10pt, a4paper]{article}

\usepackage[
	left=1.5cm,
	right=1.5cm,
	top=2cm,
	bottom=2cm,
]{geometry}

\usepackage[T1,T2A]{fontenc}
\usepackage[utf8]{inputenc}
\usepackage[english,russian]{babel}
\usepackage{amsmath}
\usepackage{amsfonts}
\usepackage{amssymb}
\usepackage{makeidx}
\usepackage{listings}
\lstset
{ %Formatting for code 
	language=haskell,
	basicstyle=\footnotesize,
	numbers=left,
	stepnumber=1,
	%frame=single,
	showstringspaces=false,
	tabsize=1,
	breaklines=true,
	breakatwhitespace=false,
}

\setlength{\parindent}{0pt}


\begin{document}
\section{Давайте поговорим о паттерн матчинге на листах}
\subsection{Введение}

\subsubsection{Определения}
{\bf Тотальная функция} - определена на всей области возможных значений

{\bf Нетотальная} - на каких-то значениях падает.

Давайте теперь рассмотрим примеры функций на списках, доступных в хаскеле, и приведем их реализации на паттрен матчинге


\subsection{Стандартные функции на листах}


\subsubsection{Функция head}
Пример - функция $head$, которая возвращает голову листа, упадет с ошибкой при передачи пустого листа. Т.е. она нетотальрная
\begin{lstlisting}[language=haskell]
head :: [a] -> a
head (x:_) = x
head _ = error "empty list"
\end{lstlisting}

\subsubsection{Функция tail}
Еще есть функция $tail$ - она вообще говоря в $Prelude$ падает на пустом списке, т.е. она нетотальна. 

Но мы можем реализовать ее посвоему. Если нет даже головы или есть элемент всего один,то результат один и тот же - пустой список. Т.е. можно определить тотальную функцию $totalTail$ 

\begin{lstlisting}[language=haskell]
totalTail :: [a] -> [a]
totalTail (_: xs) = xs
totalTail _ = []
\end{lstlisting}

\subsubsection{Функция take}
Еще есть функция $take$. Она возвращает какое-то (заданное) количество элементов

\begin{lstlisting}[language=haskell]
take :: Int -> [a] -> [a]
take 0 _ = []
take n (x: xs) = x : take (n-1) xs
take _ [] = []
\end{lstlisting}

\subsubsection{Функция map}

Преобазует $a -> b$ при помощи функции

\begin{lstlisting}[language=haskell]
map :: (a -> b) -> [a] -> [b]
map _ [] = []
map f (x: xs) = f x : map f xs
\end{lstlisting}

\subsubsection{Функция filter}

Фильтрует значения из списка

Вариант №1 (обычный)
\begin{lstlisting}[language=haskell]
filter :: (a -> Bool) -> [a] -> [a]
filter _ [] = []
filter p (x:xs) = 
    if p x 
    then x : filter p xs 
    else filter p xs
\end{lstlisting}

Вариант 2 (модный)
\begin{lstlisting}[language=haskell]
filter :: (a -> Bool) [a] -> [a]
filter _ [] = []
filter p (x: xs)
    | p x       =   x : rest
    | otherwise =       rest
    where 
        rest = filter p xs
\end{lstlisting}

\subsubsection{Функция zip}

Которая объединяет 2 списка в список пар

\begin{lstlisting}[language=haskell]
zip :: [a] -> [b] -> [(a,b)]
zip _     [] = []
zip []    _  = []
zip (x:xs) (y:ys) = (x,y): zip xs ys
\end{lstlisting}

\subsubsection{Функция zipWith}

Которая объединяет 2 списка в третий при помощи вашей любой функции

\begin{lstlisting}[language=haskell]
zipWith :: (a -> b -> c) -> [a] -> [b] -> [c]
zipWith _ _ [] = []
zipWith _ [] _ = []
zipWith f (x:xs) (y:ys) = f x y : zipWith f xs ys
\end{lstlisting}

Это упражнение можно продолжать до zipWith3, например, которая объеденит уже 3 

\begin{lstlisting}[language=haskell]
zipWith3 :: (a -> b -> c -> d) -> [a] -> [b] -> [c] -> [d]
zipWith3 _ _ _ [] = []
zipWith3 _ _ [] _ = []
zipWith3 _ [] _ _ = []
zipWith3 f (x:xs) (y:ys) (z:zs) = f x y z : zipWith3 f xs ys zs
\end{lstlisting}

А дальше можно продолжать в принципе до бесконечности


\subsubsection{Функции takeWhile и dropWhile}


takeWhile берет элементы, пока не найдет первый false, как найден -> возвращает все элементы для этого

dropWhile наоборот дропает, пока не найдет первый false, дальше возвращает весь остаток (+ тот элемент, который первый "не подошел")

В хаскеле можно определять одинковые сигнатуры вот так:

\begin{lstlisting}[language=haskell]
takeWhile, dropWhile :: (a -> Bool) -> [a] -> [a]
takeWhile _ [] = []
takeWhile p (x:xs) 
  | p x       = x: takeWhile p xs
  | otherwise = []

dropWhile _ [] = []
dropWhile p (x:xs)
  | p x       = dropWhile p xs
  | otherwise = x : xs
\end{lstlisting}

\subsubsection{Функция свертки она же fold}

Свертка - это взять все элементы некого листа и каким-то образом их объеденить (т.е. "свернуть" все элементы в 1 элемент).

Сворачивать в общем-то можно слева направо или справа налево. Для + например, разницы нет. А вот для умножения квадратных матриц очень даже влияет.

В примере ниже сверху левая свертка, снизу правая свертка.

$$((1+2) + 3) + 4$$
$$1 + (2 + (3 + 4))$$

А вот и определения сверток
\begin{lstlisting}[language=haskell]
foldl :: (acc -> b -> acc) -> acc -> [b] -> acc
foldr :: (b -> acc -> acc) -> acc -> [b] -> acc

foldl _ acc []     =  acc 
foldl f acc (x:xs) =  foldl f (f acc x) xs

foldr _ acc []     =  acc 
foldr f acc (x:xs) =  f x (foldr f acc xs)

\end{lstlisting}

На основе сверток (и на основе того, что в хаскеле из сигнатуры можно вернуть любую часть этой сигнатуры) можно очень коротко писать всякие другие классные функции. Например сумму и умножение

\begin{lstlisting}[language=haskell]
sum     = foldl (+) 0
product = foldl (*) 1 
\end{lstlisting}


\subsubsection{Бесконечный поток чисел Фибоначчи!}
Над следующим пунктом можно помедетировать. Это бесконечный список Фибоначчи. Его бесконечность нас не сильно волнует, так как в хаскеле все ленивое.
\begin{lstlisting}[language=haskell]
fibs :: [Int]
fibs = 1 : 1 : zipWith (+) fibs (tail fibs)
\end{lstlisting}


\subsubsection{Конкатенация списков}
Конкатенация делается вообще чудесно

\begin{lstlisting}[language=haskell]
(++) :: [a] -> [a] -> [a]
(x:xs) ++ ys = x : (xs ++ ys)
[]     ++ ys =            ys
\end{lstlisting}

Тут написано, что мы в первом матчинге складываем первый элемент первого списка с остатком первого списка и полный вторым списком.

А во втором матчинге, если первый список закончился - просто добавляем сзади весь второй список.

\subsection{Всякое разное}
\subsubsection{Оператор (.)}

Есть такое оператор со следующей сигнатурой
\begin{lstlisting}[language=haskell]
(.) :: (b->c) -> (a->b) -> (a->c)

-- f . g = \x -> f (g x)

pipe :: (b -> c) -> (a -> b) -> [a] -> [a]
pipe f = map f . map g 
\end{lstlisting}

Последние две строчки - это {\bf $\eta$ редукция} (читается как ета-редукция). Такая нотация еще называется "бесточечная", хотя в хаскеле здесь как раз-таки и используется точка...

Это напоминает суперпозицию функций из математики.

\subsubsection{Задачка про воду}
Есть некоторая местность. Её рельеф задан высотой столбиков. Идет дождь. Дождь прошел, а там где какие-то ямы образовались лужи. Нужно посчитать, сколько туда накапало воды...

Давайте её решим

\begin{lstlisting}[language=haskell]

r = [1, 5, 3, 8, 3, 2, 3, 7, 4]

-
-----
---**
--------
---****
--*****
---****
-------
----
\end{lstlisting}
Звездочками отмечено, где есть вода. Палочками - высота рельефа.
 
Есть такие функции $scanl$ и $snanr$. Это как $fold$, только он возвращает значение аккумулятора на каждом из шагов
\begin{lstlisting}[language=haskell]
 main = print $ scanl (+) 0 [1,2,3,4,5] -- yeild us [0,1,3,6,10,15]
\end{lstlisting}

Мы можем сделать 2 скана. 1 слева на максимум. Другой справа на максимум.
\begin{lstlisting}[language=haskell]
r = [1, 5, 3, 8, 3, 2, 3, 7, 4]
main = do
  print $ r
  print $ tail $ scanl max 0 r
  print $ init $ scanr max 0 r

[1, 5, 3, 8, 3, 2, 3, 7, 4]
[1, 5, 5, 8, 8, 8, 8, 8, 8]
[8, 8, 8, 8, 7, 7, 7, 7, 4]
\end{lstlisting}

Значение сверху - это "дно". Значения снизу (2 ряда) - это одна и вторая стенка и её "тень". По этим трем массивам видно, что нужно вычесть значение от минимума из двух стенок значение "дна" (По выводу видно, что что-то отличное от нуля у нас будет в позиции 3, а также 5,6,7).

Т.е. у нас следующий алгорим
\begin{enumerate}
	\item Вычислить границы, считая с левой стороны (т.е. запомнить последний максимум и повторять его, пока не найдем еще 1 больший максимум). Это будет что-то вроде "тени"
	\item Вычислить границы, считая с правой стороны
	\item Берем минимум от двух пар из пунктов 1 и 2
	\item Вычитаем из значение этого минимума значение в точке
	\item Складываем сумму этих результатов
\end{enumerate}

При реализации на хаскеле все будет записао как бы наоборот

\begin{lstlisting}[language=haskell]
r = [1, 5, 3, 8, 3, 2, 3, 7, 4]
main = do
   print r
   print x
   where 
       x = foldl (+) 0 ws
       ws = zipWith (-) bs r
       bs = zipWith min bl br
       bl = tail $ scanl max 0 r
       br = init $ scanr max 0 r
\end{lstlisting}


\end{document}
